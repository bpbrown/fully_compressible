\documentclass{aastex631}

\usepackage{amsthm, amsmath, amssymb}
\usepackage{latexsym,graphicx,rotating,amsmath, epsfig, natbib, graphbox}
\usepackage{listings}

\newcommand{\sol}{\odot}
\newcommand{\del}{\nabla}
\newcommand{\cross}{\times}
\newcommand{\avg}{\bar}
\renewcommand{\vec}{\boldsymbol}
\newcommand{\pomega}{\varpi}
\newcommand{\conv}{\boldsymbol}
\newcommand{\grad}{\vec{\del}}
\renewcommand{\d}{\mathrm{d}}

\newcommand{\scrD}{\mathcal{D}}
\newcommand{\scrH}{\mathcal{H}}
\newcommand{\scrR}{\mathcal{R}}
\newcommand{\scrL}{\mathcal{L}}
\newcommand{\scrS}{\mathcal{S}}
\newcommand{\scrP}{\mathcal{P}}

\newcommand{\Ma}{\mathrm{Ma}}
\newcommand{\Ra}{\mathrm{Ra}}
\newcommand{\Ek}{\mathrm{Ek}}
\renewcommand{\Pr}{\mathrm{Pr}}
\newcommand{\Pm}{\mathrm{Pm}}
\newcommand{\RoCsq}{\mathrm{Ro}_\mathrm{C}^2}
\newcommand{\RoC}{\mathrm{Ro}_\mathrm{C}}

\newcommand{\expm}{\mathrm{expm1}}

\newcommand{\dedalus}{\href{http://dedalus-project.org/}{Dedalus}}

\definecolor{codegreen}{rgb}{0,0.6,0}
\definecolor{codegray}{rgb}{0.5,0.5,0.5}
\definecolor{codepurple}{rgb}{0.58,0,0.82}
\definecolor{backcolour}{rgb}{0.95,0.95,0.92}
\definecolor{codered}{rgb}{0.6,0,0}
\definecolor{codeblue}{rgb}{0,0,0.6}

% \watermark{text}
\begin{document}
\section{Basic equations}
The fully compressible Navier-Stokes momentum equation, in standard form is:
\begin{equation}
  \partial_t \vec{u} + \vec{u}\cdot\grad\vec{u} = -\frac{\grad P}{\rho} - \grad \phi - \frac{1}{\rho}\grad \cdot (\mu E_ij).
\end{equation}
We can transform from pressure to enthalpy and entropy via thermodynamic relations:
\begin{equation}
  \d h = T \d s + (\d P)/\rho,
\end{equation}
where, without loss of generality, we'll take $\d \rightarrow \grad$.  For an ideal gas we also have:
\begin{equation}
  h = c_P T,
\end{equation}
which leads to the following momentum equation:
\begin{equation}
  \partial_t \vec{u} + \vec{u}\cdot\grad\vec{u} = -\grad (h + \phi) + h \frac{\grad s}{c_P} + \frac{1}{\rho}\grad \cdot (\mu E_{ij}).
\end{equation}

This is paired with a continuity equation:
\begin{equation}
  \partial_t \ln \rho + \vec{u} \cdot \grad \vec{u} + \grad \cdot \vec{u} = 0,
\end{equation}
an equation of state:
\begin{equation}
 \frac{\d s}{c_P} = \frac{1}{\gamma} \d \ln h - \frac{\gamma - 1}{\gamma}\d \ln \rho
\end{equation}
and and entropy equation (e.g., an energy equation).  The entropy equation comes from the second law of thermodynamics:
\begin{equation}
  \rho T \d s = \d q = \Phi - \grad \cdot \vec{F}
\end{equation}
with $\Phi$ the viscous heating term and $\vec{F}$ the fluxes of heat.  For diffusive fluxes:
\begin{equation}
  \vec{F} = -K \grad T
\end{equation}
and if we have things like internal heating, those can also be represented either as a heating rate or as an equivalent flux $\vec{F_0}$, but we'll wait on those until we have larger equilibrium system.  The final form of the entropy equation is:
\begin{equation}
  \partial_t s + \vec{u}\cdot\grad s = \frac{1}{\rho h}\left[c_P \Phi + \grad\cdot(K \grad h) - \grad\cdot\vec{F_0} \right]
\end{equation}

\newpage
\subsection{Non-dimensional form}
To obtain a set of non-dimensional equations, let $t = \tau t^*$, $s = s_c s^*$, $\vec{x} = L \vec{x^*}$, with $u_c = L/\tau$ and $\tau$ at this point un-determined.  We will also take $h = h_c h^*$ and $\rho = \rho_c \rho^*$, with starred quantities non-dimensional.

The non-dimensional momentum equation is:
\begin{equation}
  \partial_t \vec{u} + \vec{u}\cdot\grad\vec{u} =
  -\left[\frac{h_c \tau^2}{L^2}\right]\grad \left(h + \left[\frac{\phi_c}{h_c}\right]\phi\right)
  + \left[\frac{h_c \tau^2}{L^2}\right] \left[\frac{s_c}{c_P}\right]h\grad s + \left[\frac{(\mu/\rho_c) \tau}{L^2}\right]\frac{1}{\rho} \Bigg(\grad \cdot (E_{ij}) + \grad \ln \mu \cdot E_{ij}\Bigg).
\end{equation}
with entropy equation:
\begin{equation}
  \left[\frac{s_c}{c_P}\right]\Bigg(\partial_t s + \vec{u}\cdot\grad s\Bigg) =
  \left[\frac{(\mu/\rho_c)\tau}{L^2}\right]\left[\frac{L^2 \tau^2}{h_c}\right]\frac{1}{\rho h} \Phi
  + \left[\frac{K/(\rho_c c_P)\tau}{L^2}\right] \frac{1}{\rho h}\Bigg(\grad\cdot(\grad h) + \grad K \cdot \grad h \Bigg)
  - \left[\frac{F_c \tau}{L \rho_c h_c}\right]\frac{1}{\rho h}\grad\cdot\vec{F_0}.
\end{equation}
The equation of state is:
\begin{equation}
  \left[\frac{s_c}{c_P}\right] \gamma s = \ln h - (\gamma - 1)\ln \rho
\end{equation}

\subsection{Hydrostatic balance}
Let quantities with subscript zero denote the hydrostatically-balanced state.  The momentum equation gives:
\begin{equation}
  \grad h_0 =
  - \left[\frac{\phi_c}{h_c}\right]\grad\phi
  + \left[\frac{s_c}{c_P}\right]h_0\grad s_0.
\end{equation}
If we let $h_1 = h - h_0$ and $s_1 = s - s_0$, we obtain the fluctuating momentum equation:
\begin{equation}
  \partial_t \vec{u} + \vec{u}\cdot\grad\vec{u} =
  -\left[\frac{h_c \tau^2}{L^2}\right]\grad h_1
  + \left[\frac{h_c \tau^2}{L^2}\right] \left[\frac{s_c}{c_P}\right]\left(h_1\grad s_0 + h_0\grad s_1 + h_1 \grad s_1\right) + \left[\frac{(\mu/\rho_c) \tau}{L^2}\right]\frac{1}{\rho} \Bigg(\grad \cdot (E_{ij}) + \grad \ln \mu \cdot E_{ij}\Bigg).
\end{equation}

\subsection{Internal heating}
Let's consider a system that has a linearly decreasing background flux $F_0$:
\begin{equation}
  \vec{F_0} = F_c \left(1 - \epsilon z\right) \vec{\hat{z}}
\end{equation}
This decreasing flux deposits heat throughout the system.  We might assume that the characteristic flux $F_c$ is the flux carried by thermal diffusion on the adiabatic, in which case it scales as:
\begin{equation}
  F_c = -(K/c_p) \grad h_{ad}
\end{equation}
For a polytrope, with $\grad h_{ad} = \text{constant}$, the adiabatic flux is constant and disappears under the divergence, but the fall off remains.  Dimensionally,
\begin{equation}
  F_c = \frac{(K/c_p) h_c}{L}
\end{equation}
and
\begin{equation}
\left[\frac{s_c}{c_P}\right] \partial_t s \propto
\left[\frac{(K/(\rho_c c_p)) \tau}{L^2}\right]\frac{1}{\rho h} \epsilon.
\end{equation}
For simplicity, we've assumed $K$ is constant.
Under this assumption, our full entropy equation is then:
\begin{equation}
  \left[\frac{s_c}{c_P}\right]\Bigg(\partial_t s + \vec{u}\cdot\grad s\Bigg) =
  \left[\frac{(\mu/\rho_c)\tau}{L^2}\right]\left[\frac{L^2 \tau^2}{h_c}\right]\frac{1}{\rho h} \Phi
  + \left[\frac{K/(\rho_c c_P)\tau}{L^2}\right] \frac{1}{\rho h}\grad\cdot(\grad h)
  + \left[\frac{(K/(\rho_c c_p)) \tau}{L^2}\right]\frac{1}{\rho h} \epsilon.
\end{equation}

Using the results of Appendix~\ref{sec:log enthalpy diffusion}, we can re-write the entropy equation as:

\begin{equation}
  \left[\frac{s_c}{c_P}\right]\Bigg(\partial_t s + \vec{u}\cdot\grad s\Bigg) =
  \left[\frac{(\mu/\rho_c)\tau}{L^2}\right]\left[\frac{L^2 \tau^2}{h_c}\right]\frac{1}{\rho h} \Phi
  + \left[\frac{K/(\rho_c c_P)\tau}{L^2}\right] \frac{1}{\rho h}\left(\nabla^2 \theta + 2 \grad \ln h_0 \cdot \grad \theta + (\grad \theta)^2\right)
  + \left[\frac{(K/(\rho_c c_p)) \tau}{L^2}\right]\frac{1}{\rho h} \epsilon.
\end{equation}



\section{Choices}
Now it's time to finally make some choices.

\subsection{Velocity-based equations: $\tau = L/u_c$}
Let's make the exciting choice first, and try and set our timescale by our characteristic (unknown) velocity scale.

Our major non-dimensional sets are:
\begin{align}
\Ma^{2} & \equiv \left[\frac{L^2}{h_c \tau^2}\right] = \left[\frac{u_c^2}{h_c}\right] \\
\scrR & \equiv \left[\frac{L^2}{(\mu/\rho_c) \tau}\right] = \left[\frac{u_c L}{(\mu/\rho_c)}\right] \\
\scrP & \equiv \left[\frac{L^2}{K/(\rho_c c_P)\tau}\right] = \left[\frac{u_c L}{K/(\rho_c c_P)}\right] \\
\text{and} \nonumber \\
\scrP &\equiv \scrR \Pr \\
\end{align}
We have no particular guidance on the characteristic entropy scale $s_c$, so here we take
\begin{equation}
  \frac{s_c}{c_P} \equiv 1.
\end{equation}

Our non-dimensional equations under this set are:
\begin{equation}
  \partial_t \Upsilon_1 + \vec{u}\cdot\grad \Upsilon_0 + \vec{u}\cdot\grad \Upsilon_1 + \grad \cdot \vec{u} = 0
\end{equation}
\begin{equation}
  \partial_t \vec{u} + \vec{u}\cdot\grad\vec{u} =
  -\frac{1}{\Ma^2}\grad h_1
  + \frac{1}{\Ma^2} \left(h_1\grad s_0 + h_0\grad s_1 + h_1 \grad s_1\right) + \frac{1}{\scrR}\frac{1}{\rho} \grad \cdot (E_{ij})
\end{equation}
\begin{equation}
  \partial_t s + \vec{u}\cdot\grad s =
  \frac{1}{\scrR} \Ma^2  \frac{1}{\rho h} \Phi
  + \frac{1}{\scrR \Pr} \frac{1}{\rho}\left(\nabla^2 \theta + 2 \grad \ln h_0 \cdot \grad \theta + (\grad \theta)^2\right)
  + \frac{1}{\scrR \Pr} \frac{1}{\rho h} \epsilon.
\end{equation}
with equation of state:
\begin{equation}
  \gamma s_1 = \theta_1 - (\gamma - 1)\Upsilon_1
\end{equation}

\subsection{A hydrostatic and adiabatic background state}
If we assume that the hydrostatic state is an adiabatic polytrope, with $\grad s_0=0$, we can start achieving final form for our equations.
The momentum equation can be written in terms of $\theta$ and $\Upsilon$ by taking
\begin{align}
\rho &= \rho_0 \exp(\Upsilon) \\
h & = h_0 \exp(\theta) \\
h_1 &= h - h_0 = h_0(\exp(\theta)-1)
\end{align}

\begin{equation}
  \partial_t \vec{u} + \vec{u}\cdot\grad\vec{u} =
  -\frac{1}{\Ma^2} \grad \left(h_0 \left(\exp(\theta_1)-1\right)\right)
  + \frac{1}{\Ma^2} \left(h_0\grad s_1 + h_0 \left(\exp(\theta_1)-1\right)\grad s_1\right) + \frac{1}{\scrR}\frac{1}{\rho} \grad \cdot (E_{ij})
\end{equation}
Before proceeding to Dedalus-form, we note
\begin{equation}
\grad \left(h_0 \left(\exp(\theta_1)-1\right)\right) = \grad \left(h_0 \left(\exp(\theta_1)-1-\theta\right)\right) + \grad \left(h_0 \theta\right)
\end{equation}

\newpage
\section{Dedalus form}
Defining
\begin{equation}
  \expm(A) \equiv \exp(A) - 1
\end{equation}
we have continuity:
\begin{equation}
  \partial_t \Upsilon_1 + \vec{u}\cdot\grad \Upsilon_0 + \grad \cdot \vec{u} = - \vec{u}\cdot\grad \Upsilon_1
\end{equation}
momentum:
\begin{equation}
  \partial_t \vec{u}
  + \frac{1}{\Ma^2}\Bigg((\grad h_0)\theta  + h_0 \grad \theta \Bigg)
  - \frac{1}{\Ma^2} h_0 \grad s_1
  - \frac{1}{\scrR}\frac{1}{\rho} \grad \cdot (E_{ij})
  =
  -\vec{u}\cdot\grad\vec{u}
  -\frac{1}{\Ma^2} \left(\expm(\theta)-\theta\right)
  + \frac{1}{\Ma^2} \left(h_0 \expm(\theta) \grad s_1\right)
\end{equation}
entropy:
\begin{equation}
  \partial_t s
  - \frac{1}{\scrR \Pr} \frac{1}{\rho}\left(\nabla^2 \theta + 2 \grad \ln h_0 \cdot \grad \theta \right)
  =
  - \vec{u}\cdot\grad s
  + \frac{1}{\scrR} \Ma^2  \frac{1}{\rho h} \Phi
  + \frac{1}{\scrR \Pr} \frac{1}{\rho} (\grad \theta)^2
  + \frac{1}{\scrR \Pr} \frac{1}{\rho h} \epsilon.
\end{equation}
with equation of state:
\begin{equation}
  \gamma s_1 = \theta_1 - (\gamma - 1)\Upsilon_1
\end{equation}

\subsection{scale factors}
To reduce ncc bandwidth, let's scale these:
\begin{equation}
    h_0 \left(\partial_t \Upsilon_1 + \vec{u}\cdot\grad \Upsilon_0 + \grad \cdot \vec{u}\right) = - h_0 \vec{u}\cdot\grad \Upsilon_1
\end{equation}
momentum:
\begin{multline}
  \rho_0 \partial_t \vec{u}
  + \frac{1}{\Ma^2}\rho_0 h_0 \Bigg((\grad \theta_0)\theta  + \grad \theta \Bigg)
  - \frac{1}{\Ma^2} \rho_0 h_0 \grad s_1
  - \frac{1}{\scrR}\left[\frac{\rho_0}{\rho}\right] \grad \cdot (E_{ij})
  = \\
  -\rho_0 \vec{u}\cdot\grad\vec{u}
  -\frac{1}{\Ma^2} \rho_0 \left(\expm(\theta)-\theta\right)
  + \frac{1}{\Ma^2} \rho_0 h_0 \expm(\theta) \grad s_1
\end{multline}
entropy:
\begin{equation}
  \rho_0 \partial_t s
  - \frac{1}{\scrR \Pr} \left[\frac{\rho_0}{\rho}\right]\left(\nabla^2 \theta + 2 \grad \ln h_0 \cdot \grad \theta \right)
  =
  - \rho_0 \vec{u}\cdot\grad s
  + \frac{1}{\scrR} \Ma^2  \frac{\rho_0}{\rho h} \Phi
  + \frac{1}{\scrR \Pr} \left[\frac{\rho_0}{\rho}\right] (\grad \theta)^2
  + \frac{1}{\scrR \Pr} \frac{\rho_0}{\rho h} \epsilon.
\end{equation}
with equation of state:
\begin{equation}
  \gamma s_1 = \theta_1 - (\gamma - 1)\Upsilon_1
\end{equation}
If in the diffusive terms we take the square bracket $[\rho_0/\rho] = 1$ (namely, the diffusion constant only feels the background state), then this system is straightforward to timestep.  If we take a fully nonlinear density in the diffusion term $[\rho_0/\rho] = \exp(-\Upsilon)$, then we need to be much more careful in how we treat diffusivity.  We'll need to handle part of the term implicitly and part explicitly, and to do this we'll need to have the explicit part be "anti-diffusive", which means we'll need to "over-subtract" the implicit stabilized portion from the full nonlinear form.  But that's a fight for another day.

\newpage
\subsection{Just make it simple}
If we take $1/\rho = 1/\rho_0$, and $1/h = 1/h_0$ for consistency, here's the cleaned up set of equations, with continuity:
\begin{equation}
    h_0 \left(\partial_t \Upsilon_1 + \vec{u}\cdot\grad \Upsilon_0 + \grad \cdot \vec{u}\right) = - h_0 \vec{u}\cdot\grad \Upsilon_1
\end{equation}
momentum:
\begin{multline}
  \rho_0 \partial_t \vec{u}
  + \frac{1}{\Ma^2}\rho_0 h_0 \Bigg((\grad \theta_0)\theta  + \grad \theta \Bigg)
  - \frac{1}{\Ma^2} \rho_0 h_0 \grad s_1
  - \frac{1}{\scrR}\grad \cdot (E_{ij})
  = \\
  -\rho_0 \vec{u}\cdot\grad\vec{u}
  -\frac{1}{\Ma^2} \rho_0 \left(\expm(\theta)-\theta\right)
  + \frac{1}{\Ma^2} \rho_0 h_0 \expm(\theta) \grad s_1
\end{multline}
entropy:
\begin{equation}
  \rho_0 \partial_t s
  - \frac{1}{\scrR \Pr}\left(\nabla^2 \theta + 2 \grad \ln h_0 \cdot \grad \theta \right)
  =
  - \rho_0 \vec{u}\cdot\grad s
  + \frac{1}{\scrR} \Ma^2  \frac{1}{h_0} \Phi
  + \frac{1}{\scrR \Pr} (\grad \theta)^2
  + \frac{1}{\scrR \Pr} \frac{1}{h_0} \epsilon.
\end{equation}
with equation of state:
\begin{equation}
  \gamma s_1 = \theta_1 - (\gamma - 1)\Upsilon_1
\end{equation}


\newpage
\appendix
\section{Diffusion in log-enthalpy form, with $\theta = \ln h$}
\label{sec:log enthalpy diffusion}
The diffusive term can be re-written:
\begin{equation}
  \frac{1}{h}\grad\cdot(\grad h) = \nabla^2 \ln h + (\grad \ln h)^2
\end{equation}
or, taking the $h = h_0 + h_1$ decomposition:
\begin{equation}
  \nabla^2 \ln h + (\grad \ln h)^2 = \nabla^2 \ln h_0 + (\grad \ln h_0)^2 + 2\grad \ln h_0 \cdot \grad \ln h_1 + \nabla^2 \ln h_1 + (\grad \ln h_1)^2
\end{equation}
Does this hold up for a general nonlinear $1/h$?  Let's test by deriving this via a different pathway.

Inspired by our treatment of log-densities, take a log-enthalpy form, with
\begin{equation}
  \theta \equiv \ln h = \theta_0 + \theta_1
\end{equation}
and
\begin{equation}
  h_0 = h_c \exp(\theta_0)
\end{equation}
or
\begin{equation}
  h_1 = h-h_0 = h_0 \left[\exp(\theta) - 1\right]
\end{equation}


Now:
\begin{equation}
  \grad h = \grad h_0 + \grad h_1 = \grad h_0 + h_0 \exp(\theta) \grad \theta + \left[\exp(\theta) - 1\right] \grad h_0
\end{equation}
and
\begin{align}
  \grad\cdot\grad h
  &= \nabla^2 h_0 + \grad\cdot(h_0 \exp(\theta) \grad \theta + \left[\exp(\theta) - 1\right] \grad h_0) \nonumber \\
  &= \nabla^2 h_0
  + \exp(\theta) \grad h_0 \cdot \grad \theta
  + h_0 \exp(\theta) \grad \theta \cdot \grad \theta
  + h_0 \exp(\theta) \nabla^2 \theta
  + \exp(\theta) \grad\theta \cdot \grad h_0
  + \left[\exp(\theta)-1\right] \nabla^2 h_0
  \nonumber \\
  &=
  2 h_0 \exp(\theta) \grad \ln h_0 \cdot \grad \theta
  + h_0 \exp(\theta) \grad \theta \cdot \grad \theta
  + h_0 \exp(\theta) \nabla^2 \theta
  + h_0 \exp(\theta) \nabla^2 \ln h_0
  + h_0 \exp(\theta) (\grad \ln h_0)^2
\end{align}
And our overall term is:
\begin{align}
  \frac{\grad\cdot\grad h}{h}
  & = \frac{1}{h_0 \exp{\theta}} \grad\cdot\grad h \nonumber \\
  & = 2 \grad \ln h_0 \cdot \grad \theta
  + \grad \theta \cdot \grad \theta
  + \nabla^2 \theta
  + \nabla^2 \ln h_0
  + (\grad \ln h_0)^2
\end{align}
Note that the last terms came from:
\begin{equation}
  h_0 \frac{\nabla^2 h_0}{h_0} =
  h_0 \left(\nabla^2 \ln h_0 + (\grad \ln h_0)^2\right).
\end{equation}
In a polytrope,
\begin{equation}
  \nabla^2 h_0 = 0
\end{equation}
so we must also have
\begin{equation}
  \nabla^2 \ln h_0 + (\grad \ln h_0)^2 = 0
\end{equation}
and
\begin{align}
  \frac{\grad\cdot\grad h}{h}
  & = \nabla^2 \theta + 2 \grad \ln h_0 \cdot \grad \theta
  + (\grad \theta)^2,
\end{align}
which is the same result as we obtained previously.  So this is good for arbitrary decompositions (rather than us running into some nasty decomposition problem where this is only valid for $h_1 \ll h_0$ or similar nonsense).




\end{document}


\subsection{Hydrostatic balance and a polytrope}
Let quantities with subscript zero denote the hydrostatically-balanced state.  The momentum equation gives:
\begin{equation}
  \left[\frac{h_c \tau^2}{L^2}\right]\grad h_0 =
  - \left[\frac{h_c \tau^2}{L^2}\right]\left[\frac{\phi_c}{h_c}\right]\grad\phi
  + \left[\frac{h_c \tau^2}{L^2}\right] \left[\frac{s_c}{c_P}\right]h_0\grad s_0
\end{equation}
or:
\begin{equation}
  \grad \left(\ln h_0 - \left[\frac{s_c}{c_P}\right]s_0\right) =
  - \left[\frac{\phi_c}{h_c}\right]\frac{\grad\phi}{h_0}
\end{equation}
For a polytrope,
\begin{equation}
  \d \ln \rho = m \d \ln h,
\end{equation}
and with the equation of state:
\begin{equation}
  \left[\frac{s_c}{c_P}\right] s = \frac{1}{\gamma}(1 - (\gamma - 1) m) \ln h
\end{equation}
which leads to:
\begin{equation}
  \grad \left((1 - \frac{1}{\gamma}(1 - (\gamma - 1) m)) \ln h_0 \right) =
  - \left[\frac{\phi_c}{h_c}\right]\frac{\grad\phi}{h_0}
\end{equation}
or
\begin{equation}
  \left[\frac{\gamma-1}{\gamma}(1 +  m)\right] \grad \ln h_0  =
  - \left[\frac{\phi_c}{h_c}\right]\frac{\grad\phi}{h_0}
\end{equation}
and finally
\begin{equation}
  \grad h_0  =
  - \left[\frac{\gamma}{\gamma-1}\right]\left[\frac{1}{m+1}\right] \left[\frac{\phi_c}{h_c}\right] \grad\phi.
\end{equation}
If $\phi_c \grad \phi = g L


\end{document}

From Geoff Vasil's "gauged" document, the fully compressible equations are:
\begin{equation}
  \partial_t \ln \rho + \scrD_1 \cdot \vec{u} = 0
\end{equation}
\begin{equation}
  \partial_t \vec{u} + \vec{u}\cdot \del \vec{u} + \del (h + \phi) = T\del s + \vec{\scrD}_1\cdot(\nu E)
\end{equation}
\begin{equation}
  \partial_t s + \vec{u}\cdot \del s = \frac{1}{c_p T}\left[\vec{\scrD}_1 (\chi \del h) + \frac{\nu}{2}\mathrm{Tr}(E^2)\right]
\end{equation}
with
\begin{equation}
  T = \frac{h}{c_P}, \quad E = \del \vec{u} + (\del \vec{u})^\mathrm{T} - \frac{2}{3}(\del\cdot\vec{u})\mathrm{I},
\end{equation}
where
\begin{equation}
  \vec{\scrD}_g = \del + g \del \ln \rho,
\end{equation}
and linked by an ideal gas equation of state:
\begin{equation}
  \frac{\gamma}{\gamma-1} \frac{s}{c_P} - \frac{1}{\gamma - 1}\ln h + \ln \rho =0
\end{equation}

\subsection{Enthalpy all the time}
The first step is to rewrite the entropy equation to be entirely in terms of enthalphy; the diffusive term suggests a log enthalphy form:
\begin{equation}
  \frac{1}{h} \vec{\scrD}_1 \cdot (\chi \del h) =
  \del\cdot(\chi \ln h) + \chi (\del \ln h)^2 + \chi \ln \rho \cdot \ln h
  = \vec{\scrD}_1 \cdot (\chi \ln h) + \chi (\del \ln h)^2
\end{equation}
and
\begin{equation}
\frac{1}{c_P}\left(\partial_t s + \vec{u}\cdot \del s\right) = \vec{\scrD}_1 \cdot (\chi \ln h) + \chi (\del \ln h)^2 + \exp{(-\ln h)}\frac{\nu}{2}\mathrm{Tr}(E^2).
\end{equation}
We might be alarmed at trading a $1/T$ for a $\exp{(-\ln h)}$, but we shouldn't be: $1/T$ is already a band-unlimited nonlinear term, and $\mathrm{Tr}(E^2)$ is always going to be on the RHS.

The form above is best if $\chi$ is constant in time.  If we instead take $\kappa$ to be constant in space and time, then the diffusive term is:
\begin{align}
  \frac{1}{\rho T}\del\cdot\kappa \del T &= \frac{\kappa}{\rho}\frac{1}{T}\del\cdot\del T = \frac{\kappa}{\rho}\left[\nabla^2 \ln T + \left(\nabla \ln T\right)^2\right] \\
  &=
  \frac{\kappa}{\rho}\left[\nabla^2 \ln h + \left(\nabla \ln h\right)^2\right]
\end{align}
or, taking $\mu = \nu \rho$:
\begin{equation}
  \left(\partial_t \frac{s}{c_P} + \vec{u}\cdot \del \frac{s}{c_P}\right) = \kappa\exp{(-\ln \rho)}\left[\nabla^2 \ln h + \left(\nabla \ln h\right)^2\right] + \exp{(-\ln h)}\exp{(-\ln \rho)}\frac{\mu}{2}\mathrm{Tr}(E^2),
\end{equation}
which now has two exponentials in the viscous heating term, but again that term is always going to be RHS anyways.  We've also collected the s and $c_P$ terms together.

The momentum equation takes this form:
\begin{equation}
  \partial_t \vec{u} + \vec{u}\cdot \del\vec{u} + \del (\exp{(\ln h)} + \phi) = \exp{(\ln h)}\frac{\del s}{c_P} + \vec{\scrD}_1\cdot(\nu E),
\end{equation}
or
\begin{equation}
  \partial_t \vec{u} + \vec{u}\cdot \del\vec{u} + \del (\exp{(\ln h)} + \phi) = \exp{(\ln h)}\frac{\del s}{c_P} + \exp{(-\ln\rho)}\mu \del\cdot(E),
\end{equation}
for constant $\mu$.
Clearly we're going to have to be a bit careful with our now nonlinear pressure gradient term (buoyancy has also changed from a quadratic nonlinearity to a band-unlimited nonlinearity).  To make progress, let's consider equilibria.

\subsection{Hydrostatic equilibrium in a polytrope}
If we assert hydrostatic equilbrium, we have:
\begin{equation}
  \del (h + \phi) = h\frac{\del s}{c_P}
\end{equation}
or
\begin{equation}
  \del \ln h  - \frac{\del s}{c_P} = - \frac{\del \phi}{h}
\end{equation}
For an ideal gas we take
\begin{equation}
  \frac{\del s}{c_P} = \frac{1}{\gamma}\del \ln h - \frac{\gamma-1}{\gamma}\del \ln \rho
\end{equation}
and for a polytrope, we take
\begin{equation}
  \ln \rho = m \ln h
\end{equation}
in place of the thermal equation.

If we take $L = R T_c/\phi_c$, or the density scale height of the equivalent isothermal atmosphere at the characteristic level (say the bottom of the atmosphere), take constant gravity, and take $s_c = c_P$, then we get this non-dimensional form:
\begin{equation}
  \del \ln h  - \del s = - \frac{\gamma-1}{\gamma}\exp{(-\ln h)} \vec{\hat{z}}
\end{equation}
where the $c_P$ in the enthalpy has been factored to pull out the $R$ separately from the $(\gamma-1)/\gamma$

You know, I don't think there is actually any lengthscale information here.  What we have is $R T_c/\phi_c = 1$, because if L is the characteristic lengthscale for $\del \phi$, then the L's cancel out.  Does that somehow imply that $L=H_\rho$?  Or that $L=L_z$, the depth of the domain.  Alternatively, the correct scale is $\del \phi = \vec{g}$ and then we have a lengthscale $ L = R T_c/g_c = H_\rho$.  Very confusing.

I wonder also if there's anything in this ratio:
\begin{equation}
  \frac{\phi_c}{h_c}~\text{or}~L=\frac{g}{h_c}
\end{equation}
I guess that's an adiabatic enthalphy scale height?
(based on:)
\begin{equation}
  \del \ln h = - \frac{\del \phi}{h}, \quad\text{with} \quad \del s = 0.
\end{equation}
For now we stick to a density scaleheight system, just because.

Here's some code that solves this above problem:
\lstset{language=Python,
  backgroundcolor=\color{backcolour},
  commentstyle=\color{codered},
  keywordstyle=\color{codepurple},
  numberstyle=\tiny\color{codegray},
  stringstyle=\color{codepurple},
  frame=lines,
  breakatwhitespace=false,
  breaklines=true,
  captionpos=b,
  keepspaces=true,
  numbers=left,
  numbersep=5pt,
  showspaces=false,
  showstringspaces=false,
  showtabs=false,
  tabsize=2,
  inputencoding=utf8,
  extendedchars=true,
  literate={Υ}{{$\Upsilon$}}1 {θ}{{$\theta$}}1 {τ}{{$\tau$}}1 {φ}{{$\phi$}}1 {γ}{{$\gamma$}}1,
  }
\begin{lstlisting}
HS_problem = problems.NLBVP([Υ, θ, S, τθ])
HS_problem.add_equation((grad(θ) - grad(S) + τθ*P1 ,
                         -1*exp(-θ)*grad_φ*ez))
HS_problem.add_equation((Υ - m*θ, 0))
HS_problem.add_equation((S - 1/γ*θ+(γ-1)/γ*Υ, 0))
HS_problem.add_equation((θ(z=0),0))
\end{lstlisting}

\subsection{Exact solutions to polytropes}
The exact solution to a polytrope is:
\begin{equation}
  h(z) = h(z=0) + h_z z
\end{equation}
where
\begin{equation}
  h(z=0) =
  \begin{cases}
    1, \text{if}\ z_c=0\\
    1+L_z h_z, \text{if}\ z_c=L_z
  \end{cases}
\end{equation}
and if $L = (h/g)(z=z_c) = H_h(z=z_c)$:
\begin{equation}
\del_\phi = 1 \quad\text{and}\quad
h_z = -\frac{\gamma}{\gamma-1}\frac{1}{(m+1)}
\end{equation}
but if $L = (RT/g)(z=z_c) = H_\rho(z=z_c)$:
\begin{equation}
\del_\phi = \frac{(\gamma-1)}{\gamma} \quad\text{and}\quad
h_z = -\frac{1}{(m+1)}
\end{equation}


\section{A thermal equilibrum state}
Let's now assert a base state, assuming hydrostatic equilibrium and a thermal equilibrium.

Thermal equilibrium for this system is:
\begin{equation}
\left(\partial_t \frac{s_0}{c_P}\right) = \kappa\exp{(-\ln \rho_0)}\left[\nabla^2 \ln h_0 + \left(\nabla \ln h_0\right)^2\right] = 0
\end{equation}
or
\begin{equation}
\nabla^2 \ln h_0 + \left(\nabla \ln h_0\right)^2 = 0
\end{equation}

Recognizing that:
\begin{equation}
  \exp{(\ln h)} = \exp{(\ln h_0)}\exp{(\ln h_1)} = h_0 \exp{(\ln h_1)}
\end{equation}
and that:
\begin{equation}
  \del \left[h_0 \exp{(\ln h_1)} + \phi\right] = \del \left[h_0 (\exp{(\ln h_1)}-1) +h_0 + \phi\right]
\end{equation}
combined with:
\begin{align}
    h \frac{\del s}{c_P}
    &= h_0 \exp{(\ln h_1)}\frac{\del(s_0+s_1)}{c_P} \\
    &= h_0 \frac{\del s_0}{c_P}
    + h_0 \frac{\del s_1}{c_P}
    + h_0 \left[\exp{(\ln h_1)}-1\right]\frac{\del s_0}{c_P}
    + h_0 \left[\exp{(\ln h_1)}-1\right]\frac{\del s_1}{c_P}
\end{align}
with hydrostatic equilbrium:
\begin{equation}
  \del \left[h_0 + \phi\right] = h_0 \frac{\del s_0}{c_P}
\end{equation}
we have our pressure/buoyancy balance:
\begin{equation}
  \del\left[h_0\left(\exp{(\ln h_1)}-1\right)\right]
  = h_0 \frac{\del s_1}{c_P}
  + h_0 \left[\exp{(\ln h_1)}-1\right]\frac{\del s_0}{c_P}
  + h_0 \left[\exp{(\ln h_1)}-1\right]\frac{\del s_1}{c_P}.
\end{equation}

Let's work a bit further to pull out all linear-like terms.
First, let's take $\theta = \ln h_1$ and absorb $c_P$ into $s$:
\begin{equation}
  \del\left[h_0\left(\exp{(\theta)}-1\right)\right]
  = h_0 \del s_1
  + h_0 \left[\exp{(\theta)}-1\right] \del s_0
  + h_0 \left[\exp{(\theta)}-1\right] \del s_1.
\end{equation}
Now we add/subtract $\theta$ terms to de-stiffen $\exp{\theta}$ when linear terms can be isolated:
\begin{align}
  \del\left[h_0 \theta\right] + \del\left[h_0\left(\exp{(\theta)} -1 -\theta \right)\right]
  = h_0 \del s_1
  + h_0 \theta \del s_0
  + h_0 \left[\exp{(\theta)} -1 -\theta\right] \del s_0
  + h_0 \left[\exp{(\theta)} -1 \right] \del s_1.
\end{align}
Now we collect linear terms on the left, and nonlinear terms on the right:
\begin{equation}
  \del\left[h_0 \theta\right] - h_0 \del s_1 - (h_0 \del s_0)\theta
  =
  - \del\left[h_0\left(\exp{(\theta)} -1 -\theta \right)\right]
  + h_0 \left[\exp{(\theta)} -1 -\theta\right] \del s_0
  + h_0 \left[\exp{(\theta)} -1 \right] \del s_1.
\end{equation}

Collectively, this leads to a momentum equation with:
\begin{multline}
  \partial_t \vec{u}  +
  \del\left[h_0 \theta\right] - (h_0 \del s_0)\theta - h_0 \del s_1
  = \\
  - \vec{u}\cdot \del\vec{u}
  - \del\left[h_0\left(\exp{(\theta)} -1 -\theta \right)\right]
  + (h_0 \del s_0)\left[\exp{(\theta)} -1 -\theta\right]
  + h_0 \left[\exp{(\theta)} -1 \right] \del s_1
  + \frac{\mu}{\rho}\vec{\scrD}_1\cdot(E)
\end{multline}


\emph{A note on computation:} the $\left(\exp{(\ln h_1)}-1\right)$ term should be computed using \verb+numpy.expm1(x)+ rather than computing \verb+numpy.exp(x)+ and then subtracting 1, especially when $x \sim \mathrm{Ma}^2 \ll 1$ (numerical convergence issues).

\textbf{To do: add np.expm1 to the UnaryGridFunction list.}

With the $\expm$ operator:
\begin{multline}
  \partial_t \vec{u}  +
  \del\left[h_0 \theta\right] - (h_0 \del s_0)\theta - h_0 \del s_1
  = \\
  - \vec{u}\cdot \del\vec{u}
  - \del\left[h_0\left(\expm{(\theta)} -\theta \right)\right]
  + (h_0 \del s_0)\left[\expm{(\theta)} -\theta\right]
  + h_0 \left[\expm{(\theta)}\right] \del s_1
  + \frac{\mu}{\rho}\vec{\scrD}_1\cdot(E)
\end{multline}
We've moved various linear terms to the LHS and subtracted them off the corresponding nonlinearity on the RHS, which may help in stability especially at low Mach numbers.


\subsection{Non-dimensional form}
If we non-dimensionalize on a characteristic velocity $u_c$, then we're seeing the following non-dimensional parameter:
\begin{equation}
  \frac{h_c}{u_c^2} =
  \left(\frac{\gamma}{\gamma-1}\right)\left(\frac{R T}{u_c^2}\right) =
  \left(\frac{\gamma}{\gamma-1}\right) \mathrm{Ma}^2 =
  c_P \mathrm{Ma}^2
\end{equation}
for an isothermal Mach number $\mathrm{Ma}$.

If we also take a characteristic lengthscale $L_c = H_\rho$, and resulting timescale $\tau = L_c/u_c$, our time-derivative terms are of dimensional scale $u_c/\tau \partial_t = u_c^2/L_c \partial_t$.
This suggests:
\begin{align}
  \partial_t \vec{u}  +
  c_P \mathrm{Ma}^2 \del\left[h_0 \theta\right] &- c_P \mathrm{Ma}^2 (h_0 \del s_0)\theta - c_P \mathrm{Ma}^2 h_0 \del s_1
  = \nonumber \\
  &- \vec{u}\cdot \del\vec{u}
  - c_P \mathrm{Ma}^2 \del\left[h_0\left(\expm{(\theta)} \nonumber -\theta \right)\right] \nonumber \\
  & + c_P \mathrm{Ma}^2 (h_0 \del s_0)\left[\expm{(\theta)} -\theta\right]
  + c_P \mathrm{Ma}^2 h_0 \left[\expm{(\theta)}\right] \del s_1 \nonumber \\
  &+ \scrR \frac{1}{\rho_0}\exp{(-\ln \rho_1)}\vec{\scrD}_1\cdot(E).
\end{align}
The last term is
\begin{equation}
  \scrR = \frac{\mu}{\rho_c u_c L_c}.
\end{equation}
If we play the same linearization games with the diffusion term:
\begin{equation}
\scrR \frac{1}{\rho_0}\exp{(-\ln \rho_1)}\vec{\scrD}_1\cdot(E)
= \scrR \frac{1}{\rho_0}\vec{\scrD}_1\cdot(E)
\scrR \frac{1}{\rho_0}\left[\exp{(-\ln \rho_1)}-1\right]\vec{\scrD}_1\cdot(E).
\end{equation}
Taking $\Upsilon = \ln \rho_1$:
\begin{align}
  \partial_t \vec{u}  +
  c_P \mathrm{Ma}^2 \del\left[h_0 \theta\right] &- c_P \mathrm{Ma}^2 (h_0 \del s_0)\theta - c_P \mathrm{Ma}^2 h_0 \del s_1
  - \scrR \frac{1}{\rho_0}\vec{\scrD}_1\cdot(E) = \nonumber \\
  &- \vec{u}\cdot \del\vec{u}
  - c_P \mathrm{Ma}^2 \del\left[h_0\left(\expm{(\theta)} \nonumber -\theta \right)\right] \nonumber \\
  & + c_P \mathrm{Ma}^2 (h_0 \del s_0)\left[\expm{(\theta)} -\theta\right]
  + c_P \mathrm{Ma}^2 h_0 \left[\expm{(\theta)}\right] \del s_1 \nonumber \\
  &+ \scrR \frac{1}{\rho_0}\left[\expm{(\Upsilon)}\right]\vec{\scrD}_1\cdot(E).
\end{align}

\end{document}

\newpage

Collectively, this leads to a momentum equation with:
\begin{equation}
  \partial_t \vec{u} + \vec{u}\cdot \del\vec{u} + \del\left[h_0\left(\exp{(\ln h_1)}-1\right)\right] = \frac{h_0}{c_P}\exp{(\ln h_1)}\del s + \frac{h_0}{c_P}\left(\exp{(\ln h_1)}-1\right)\del s_0 + \frac{\mu}{\rho}\vec{\scrD}_1\cdot(E),
\end{equation}
or
\begin{equation}
  \partial_t \vec{u} + \vec{u}\cdot \del\vec{u} + \del\left[h_0\left(\exp{(\ln h_1)}-1\right)\right] -h_0 \ln h_1 \frac{\del s_0}{c_P} = h_0\exp{(\ln h_1)}\frac{\del s}{c_P} + h_0\left(\exp{(\ln h_1)}-1-\ln h_1\right)\frac{\del s_0}{c_P} + \frac{\mu}{\rho}\vec{\scrD}_1\cdot(E),
\end{equation}
or, going to $\theta = \ln h_1$ and absorbing $c_P$ into $s$:
\begin{multline}
  \partial_t \vec{u}  + \del\left[h_0\theta\right] - h_0\del s - (h_0 \del s_0) \theta  = \\
  - \vec{u}\cdot \del\vec{u} - \del\left[h_0\left(\exp{(\theta)}-1-\theta\right)\right] + h_0\left(\exp{(\theta)}-1\right)\del s
  + h_0\left(\exp{(\theta)}-1-\theta\right)\del s_0 + \frac{\mu}{\rho}\vec{\scrD}_1\cdot(E).
\end{multline}
We've moved various linear terms to the LHS and subtracted them off the corresponding nonlinearity on the RHS, which may help in stability especially at low Mach numbers.

\subsection{non-dimensional form}
If we non-dimensionalize on a characteristic velocity $u_c$, then we're seeing the following non-dimensional parameter:
\begin{equation}
  \frac{h_c}{u_c^2} =
  \left(\frac{\gamma}{\gamma-1}\right)\left(\frac{R T}{u_c^2}\right) =
  \left(\frac{\gamma}{\gamma-1}\right) \mathrm{Ma}^2 =
  c_P \mathrm{Ma}^2
\end{equation}
for an isothermal Mach number $\mathrm{Ma}$.  This suggests:
\begin{align}
\partial_t \vec{u} + &c_P \mathrm{Ma}^2 \left(\del\left[h_0\theta\right] -h_0\del s - (h_0 \del s_0) \theta \right)  =
 -  \vec{u}\cdot \del\vec{u} + \frac{\mu}{\rho}\vec{\scrD}_1\cdot(E) \\
&\phantom{=} - c_P \mathrm{Ma}^2\left(\del\left[h_0\left(\exp{(\theta)}-1-\theta\right)\right] - h_0\left(\exp{(\theta)}-1\right)\del s
- h_0\left(\exp{(\theta)}-1-\theta\right)\del s_0 \right) \nonumber
\end{align}

\emph{A note on computation:} the $\left(\exp{(\ln h_1)}-1\right)$ term should be computed using \verb+numpy.expm1(x)+ rather than computing \verb+numpy.exp(x)+ and then subtracting 1, especially when $x \sim \mathrm{Ma}^2 \ll 1$ (numerical convergence issues).

\textbf{To do: add np.expm1 to the UnaryGridFunction list.}

With the $\expm$ operator:
\begin{align}
\partial_t \vec{u} + &c_P \mathrm{Ma}^2 \left(\del\left[h_0\theta\right] -h_0\del s - (h_0 \del s_0) \theta \right)  =
-  \vec{u}\cdot \del\vec{u} + \frac{\mu}{\rho}\vec{\scrD}_1\cdot(E) \\
&\phantom{=} - c_P \mathrm{Ma}^2\left(\del\left[h_0\left(\expm{(\theta)}-\theta\right)\right] - h_0\expm{(\theta)}\del s
- h_0\left(\expm{(\theta)}-\theta\right)\del s_0 \right) \nonumber
\end{align}


\section{Here we try an adiabatic state}

\subsection{Some kind of equilibrium}
Let's now assert a base state, assuming hydrostatic equilibrium and an adiabatic profile.  This differs from assuming thermal equilibrium.
Let:
\begin{equation}
  \del s_0 = 0
\end{equation}
and
\begin{equation}
  \del(h_{0,0}\exp{\ln h_0} + \phi) = 0
\end{equation}
this means:
\begin{equation}
  h_{0,0}\exp{\ln h_0} + \phi = h_0 + \phi = \scrH
\end{equation}
for some gauge constant $\scrH$.

\subsection{Momentum equation about HS eq}

What's the momentum equation look like for fluctuations about this hydrostatic, adiabatic equilibrium?

Recognizing that:
\begin{equation}
  \exp{(\ln h)} = \exp{(\ln h_0)}\exp{(\ln h_1)} = h_0 \exp{(\ln h_1)}
\end{equation}
and that:
\begin{equation}
  \del \left[h_0 \exp{(\ln_1)} + \phi\right] = \del\left[h_0\left(\exp{(\ln h_1)}-1\right)\right]
\end{equation}
we have:
\begin{equation}
  \partial_t \vec{u} + \vec{u}\cdot \del\vec{u} + \del\left[h_0\left(\exp{(\ln h_1)}-1\right)\right] = \frac{h_0}{c_P}\exp{(\ln h_1)}\del s + \vec{\scrD}_1\cdot(\nu E),
\end{equation}
If we non-dimensionalize on a characteristic velocity $u_c$, then we're seeing the following non-dimensional parameter:
\begin{equation}
  \frac{h_0}{u_c^2} = \left(\frac{\gamma}{\gamma-1}\right)\left(\frac{R T}{u_c^2}\right) = \left(\frac{\gamma}{\gamma-1}\right) \mathrm{Ma}^2
\end{equation}
for an isothermal Mach number $\mathrm{Ma}$.  This suggests:
\begin{equation}
  \partial_t \vec{u} + \vec{u}\cdot \del\vec{u} + \left(\frac{\gamma}{\gamma-1}\right) \mathrm{Ma}^2 \del\left[h_0\left(\exp{(\ln h_1)}-1\right)\right] = \mathrm{Ma}^2 h_0 \exp{(\ln h_1)}\del s + \vec{\scrD}_1\cdot(\nu E),
\end{equation}
with $h_0$ now scaled to some reference value.

\emph{A note on computation:} the $\left(\exp{(\ln h_1)}-1\right)$ term should be computed using \verb+numpy.expm1(x)+ rather than computing \verb+numpy.exp(x)+ and then subtracting 1, especially when $x \sim \mathrm{Ma}^2 \ll 1$ (numerical convergence issues).

\textbf{To do: add np.expm1 to the UnaryGridFunction list.}

There's one more trick that's likely useful for low-Mach settings:
\begin{equation}
  \partial_t \vec{u} + \vec{u}\cdot \del\vec{u} + \left(\frac{\gamma}{\gamma-1}\right) \mathrm{Ma}^2 \del\left[h_0 \ln h_1 \right] = \mathrm{Ma}^2 h_0 \exp{(\ln h_1)}\del s -
  \left(\frac{\gamma}{\gamma-1}\right) \mathrm{Ma}^2 \del\left[h_0\left(\exp{(\ln h_1)}-1-\ln h_1\right)\right] + \vec{\scrD}_1\cdot(\nu E),
\end{equation}
or
\begin{equation}
  \partial_t \vec{u} + \vec{u}\cdot \del\vec{u} + \left(\frac{\gamma}{\gamma-1}\right) \mathrm{Ma}^2 \del\left[h_0 \ln h_1 \right] = \mathrm{Ma}^2 h_0 \exp{(\ln h_1)}\del s -
  \left(\frac{\gamma}{\gamma-1}\right) \mathrm{Ma}^2 \del\left[h_0\left(\mathrm{expm1}{(\ln h_1)}-\ln h_1\right)\right] + \vec{\scrD}_1\cdot(\nu E),
\end{equation}
where we've just shifted a likely wave-like linear term to the LHS.  Not clear there is a similar one in the quadratic thermal nonlinearity, but that's also because we assumed $\del s_0 = 0$, and there are no gravity waves in that basic state (just acoustic).  That's actually very interesting.

\subsection{Making a super-adiabatic atmosphere}
If the zero state has $\del s_0 = 0$, then the superadiabaticity must be in the initial conditions.  Here's what that looks like.

The hydrostatically balanced temperature profile is:
\begin{equation}
  T = \frac{g}{c_P} \ldots
\end{equation}

\subsection{Thermal considerations}



The lack of thermal equilibrium implies a constant heating source Q:
\begin{equation}
  Q = \vec{\scrD}_1 \cdot (\chi \ln h_0) + \chi (\del \ln h_0)^2
\end{equation}

The equation set becomes:
\begin{equation}
  \partial_t \vec{u} + \vec{u}\cdot \vec{u} + \del (h_0\left[\exp{(\ln h_1)}-1\right]) = \frac{1}{c_P}h_0(\exp{(\ln h_1)})\del s_1 + \vec{\scrD}_1\cdot(\nu E),
\end{equation}
or
\begin{equation}
  \partial_t \vec{u} + \vec{u}\cdot \vec{u} + \del (h_0 \ln h_1) - \frac{1}{c_P}h_0 \del s_1 = -\del (h_0\left[\exp{(\ln h_1)}-1-\ln h_1 \right]) + \frac{1}{c_P}h_0(\exp{(\ln h_1)} - 1)\del s_1 + \vec{\scrD}_1\cdot(\nu E),
\end{equation}
where we have de-stiffened both the nonlinear pressure gradient term and the nonlinear buoyancy term.
The viscous term is:
\begin{equation}
  \vec{\scrD}_1\cdot(\nu E) = (\del + \del \ln \rho_0 + \del \ln \rho_1) \cdot(\nu E)
\end{equation}
so:
\begin{equation}
  \partial_t \vec{u} + \vec{u}\cdot \vec{u} + \del (h_0 \ln h_1) - \frac{1}{c_P}h_0 \del s_1 - \vec{\scrD}_{1,0} \cdot(\nu E) = -\del (h_0\left[\exp{(\ln h_1)}-1-\ln h_1 \right]) + \frac{1}{c_P}h_0(\exp{(\ln h_1)} - 1)\del s_1 + \del \ln \rho_1\cdot(\nu E),
\end{equation}
We need some better notation.  Let $\Theta = \ln h$ and $\Upsilon = \ln \rho$.
\begin{equation}
  \partial_t \vec{u} + \vec{u}\cdot \vec{u} + \del (h_0 \Theta_1) - \frac{1}{c_P}h_0 \del s_1 - \vec{\scrD}_{1,0} \cdot(\nu E) = -\del (h_0\left[\exp{\Theta_1}-1-\Theta_1 \right]) + \frac{1}{c_P}h_0(\exp{\Theta_1} - 1)\del s_1 + \del \Upsilon_1\cdot(\nu E),
\end{equation}

For the entropy equation, we need to decompose the RHS:
\begin{align}
\vec{\scrD}_1 \cdot (\chi \Theta) + \chi (\del \Theta)^2 &=
\vec{\scrD}_1 \cdot (\chi \Theta_0) + \chi (\del \Theta_0)^2
+ \vec{\scrD}_1 \cdot (\chi \Theta_1) + \chi (\del \Theta_1)^2
+ 2 \chi (\del \Theta_0\cdot \del \Theta_1) \\
& = \vec{\scrD}_{1,0} \cdot (\chi \Theta_0) + \chi (\del \Theta_0)^2 \\
& \phantom{=} + \del \Upsilon_1 \cdot(\chi \Theta_0)
+ \vec{\scrD}_{1,0} \cdot (\chi \Theta_1) \\
& \phantom{=} + \del \Upsilon_1 \cdot(\chi \Theta_1)+ \chi (\del \Theta_1)^2
 + 2 \chi (\del \Theta_0\cdot \del \Theta_1) \\
& = Q + \del \Upsilon_1 \cdot(\chi \Theta_1)+ \chi (\del \Theta_1)^2
+ 2 \chi (\del \Theta_0\cdot \del \Theta_1) \\
& \phantom{=} + \del \Upsilon_1 \cdot(\chi \Theta_0)
+ \vec{\scrD}_{1,0} \cdot (\chi \Theta_1)
\end{align}
the entropy equation is:
\begin{equation}
\frac{1}{c_P}\left(\partial_t s_1 + \vec{u}\cdot \del s_1\right) =
Q +
 \vec{\scrD}_1 \cdot (\chi \ln h) + \chi (\del \ln h)^2 + \exp{(-\ln h)}\frac{\nu}{2}\mathrm{Tr}(E^2).
\end{equation}

Our variables are now:
\begin{equation}
\ln h = \ln h_0 + \ln h_1, \quad ln \rho = \ln \rho_0 + \ln \rho_1, \quad s = s_0 + s_1.
\end{equation}

Thermal equilibrium means:




\end{document}
