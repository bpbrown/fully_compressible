\documentclass{aastex62}

\usepackage{amsthm, amsmath, amssymb}
\usepackage{latexsym,graphicx,rotating,amsmath, epsfig, natbib, graphbox}

\newcommand{\sol}{\odot}
\newcommand{\del}{\nabla}
\newcommand{\cross}{\times}
\newcommand{\avg}{\bar}
\renewcommand{\vec}{\boldsymbol}
\newcommand{\pomega}{\varpi}
\newcommand{\conv}{\boldsymbol}

\newcommand{\scrD}{\mathcal{D}}
\newcommand{\scrH}{\mathcal{H}}
\newcommand{\scrR}{\mathcal{R}}
\newcommand{\scrL}{\mathcal{L}}
\newcommand{\scrS}{\mathcal{S}}

\newcommand{\Ra}{\mathrm{Ra}}
\newcommand{\Ek}{\mathrm{Ek}}
\renewcommand{\Pr}{\mathrm{Pr}}
\newcommand{\Pm}{\mathrm{Pm}}
\newcommand{\RoCsq}{\mathrm{Ro}_\mathrm{C}^2}
\newcommand{\RoC}{\mathrm{Ro}_\mathrm{C}}

\newcommand{\dedalus}{\href{http://dedalus-project.org/}{Dedalus}}

% \watermark{text}
\begin{document}
\section{Basic equations}
From Geoff Vasil's "gauged" document, the fully compressible equations are:
\begin{equation}
  \partial_t \ln \rho + \scrD_1 \cdot \vec{u} = 0
\end{equation}
\begin{equation}
  \partial_t \vec{u} + \vec{u}\cdot \vec{u} + \del (h + \phi) = T\del s + \vec{\scrD}_1\cdot(\nu E)
\end{equation}
\begin{equation}
  \partial_t s + \vec{u}\cdot \del s = \frac{1}{T}\left[\vec{\scrD}_1 (\chi \del h) + \frac{\nu}{2}\mathrm{Tr}(E^2)\right]
\end{equation}
with
\begin{equation}
  T = \frac{h}{c_P}, \quad E = \del \vec{u} + (\del \vec{u})^\mathrm{T} - \frac{2}{3}(\del\cdot\vec{u})\mathrm{I},
\end{equation}
where
\begin{equation}
  \vec{\scrD}_g = \del + g \del \ln \rho,
\end{equation}
and linked by an ideal gas equation of state:
\begin{equation}
  \frac{\gamma}{\gamma-1} \frac{s}{c_P} - \frac{1}{\gamma - 1}\ln h + \ln \rho =0
\end{equation}

\subsection{Enthalpy all the time}
The first step is to rewrite the entropy equation to be entirely in terms of enthalphy; the diffusive term suggests a log enthalphy form:
\begin{equation}
  \frac{1}{h} \vec{\scrD}_1 \cdot (\chi \del h) =
  \del\cdot(\chi \ln h) + \chi (\del \ln h)^2 + \chi \ln \rho \cdot \ln h
  = \vec{\scrD}_1 \cdot (\chi \ln h) + \chi (\del \ln h)^2
\end{equation}
and
\begin{equation}
\frac{1}{c_P}\left(\partial_t s + \vec{u}\cdot \del s\right) = \vec{\scrD}_1 \cdot (\chi \ln h) + \chi (\del \ln h)^2 + \exp{(-\ln h)}\frac{\nu}{2}\mathrm{Tr}(E^2).
\end{equation}
We might be alarmed at trading a $1/T$ for a $\exp{(-\ln h)}$, but we shouldn't be: $1/T$ is already a band-unlimited nonlinear term, and $\mathrm{Tr}(E^2)$ is always going to be on the RHS.

The momentum equation takes this form:
\begin{equation}
  \partial_t \vec{u} + \vec{u}\cdot \vec{u} + \del (\exp{(\ln h)} + \phi) = \frac{1}{c_P}\exp{(\ln h)}\del s + \vec{\scrD}_1\cdot(\nu E),
\end{equation}
so clearly we're going to have to be a bit careful with our now nonlinear pressure gradient term (buoyancy has also changed from a quadratic nonlinearity to a band-unlimited nonlinearity).  To make progress, let's consider equilibria.

\subsection{Some kind of equilibrium}
Let's now assert a base state, assuming hydrostatic equilibrium and an adiabatic profile.  This differs from assuming thermal equilibrium.
Let:
\begin{equation}
  \del s_0 = 0
\end{equation}
and
\begin{equation}
  \del(\exp{\ln h_0} + \phi) = 0
\end{equation}
this means:
\begin{equation}
  \exp{\ln h_0} + \phi = h_0 + \phi = \scrH
\end{equation}
for some guage constant $\scrH$.

The lack of thermal equilibrium implies a constant heating source Q:
\begin{equation}
  Q = \vec{\scrD}_1 \cdot (\chi \ln h_0) + \chi (\del \ln h_0)^2
\end{equation}

The equation set becomes:
\begin{equation}
  \partial_t \vec{u} + \vec{u}\cdot \vec{u} + \del (h_0\left[\exp{(\ln h_1)}-1\right]) = \frac{1}{c_P}h_0(\exp{(\ln h_1)})\del s_1 + \vec{\scrD}_1\cdot(\nu E),
\end{equation}
or
\begin{equation}
  \partial_t \vec{u} + \vec{u}\cdot \vec{u} + \del (h_0 \ln h_1) - \frac{1}{c_P}h_0 \del s_1 = -\del (h_0\left[\exp{(\ln h_1)}-1-\ln h_1 \right]) + \frac{1}{c_P}h_0(\exp{(\ln h_1)} - 1)\del s_1 + \vec{\scrD}_1\cdot(\nu E),
\end{equation}
where we have de-stiffened both the nonlinear pressure gradient term and the nonlinear buoyancy term.
The viscous term is:
\begin{equation}
  \vec{\scrD}_1\cdot(\nu E) = (\del + \del \ln \rho_0 + \del \ln \rho_1) \cdot(\nu E)
\end{equation}
so:
\begin{equation}
  \partial_t \vec{u} + \vec{u}\cdot \vec{u} + \del (h_0 \ln h_1) - \frac{1}{c_P}h_0 \del s_1 - \vec{\scrD}_{1,0} \cdot(\nu E) = -\del (h_0\left[\exp{(\ln h_1)}-1-\ln h_1 \right]) + \frac{1}{c_P}h_0(\exp{(\ln h_1)} - 1)\del s_1 + \del \ln \rho_1\cdot(\nu E),
\end{equation}
We need some better notation.  Let $\Theta = \ln h$ and $\Upsilon = \ln \rho$.
\begin{equation}
  \partial_t \vec{u} + \vec{u}\cdot \vec{u} + \del (h_0 \Theta_1) - \frac{1}{c_P}h_0 \del s_1 - \vec{\scrD}_{1,0} \cdot(\nu E) = -\del (h_0\left[\exp{\Theta_1}-1-\Theta_1 \right]) + \frac{1}{c_P}h_0(\exp{\Theta_1} - 1)\del s_1 + \del \Upsilon_1\cdot(\nu E),
\end{equation}

For the entropy equation, we need to decompose the RHS:
\begin{align}
\vec{\scrD}_1 \cdot (\chi \Theta) + \chi (\del \Theta)^2 &=
\vec{\scrD}_1 \cdot (\chi \Theta_0) + \chi (\del \Theta_0)^2
+ \vec{\scrD}_1 \cdot (\chi \Theta_1) + \chi (\del \Theta_1)^2
+ 2 \chi (\del \Theta_0\cdot \del \Theta_1) \\
& = \vec{\scrD}_{1,0} \cdot (\chi \Theta_0) + \chi (\del \Theta_0)^2 \\
& \phantom{=} + \del \Upsilon_1 \cdot(\chi \Theta_0)
+ \vec{\scrD}_{1,0} \cdot (\chi \Theta_1) \\
& \phantom{=} + \del \Upsilon_1 \cdot(\chi \Theta_1)+ \chi (\del \Theta_1)^2
 + 2 \chi (\del \Theta_0\cdot \del \Theta_1) \\
& = Q + \del \Upsilon_1 \cdot(\chi \Theta_1)+ \chi (\del \Theta_1)^2
+ 2 \chi (\del \Theta_0\cdot \del \Theta_1) \\
& \phantom{=} + \del \Upsilon_1 \cdot(\chi \Theta_0)
+ \vec{\scrD}_{1,0} \cdot (\chi \Theta_1)
\end{align}
the entropy equation is:
\begin{equation}
\frac{1}{c_P}\left(\partial_t s_1 + \vec{u}\cdot \del s_1\right) =
Q +
 \vec{\scrD}_1 \cdot (\chi \ln h) + \chi (\del \ln h)^2 + \exp{(-\ln h)}\frac{\nu}{2}\mathrm{Tr}(E^2).
\end{equation}

Our variables are now:
\begin{equation}
\ln h = \ln h_0 + \ln h_1, \quad ln \rho = \ln \rho_0 + \ln \rho_1, \quad s = s_0 + s_1.
\end{equation}

Thermal equilibrium means:




\end{document}
